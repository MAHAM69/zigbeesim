\chapter{Záver}
\indent\indent V práci sme prezentovali simulačný model sietí ZigBee a IEEE 802.15.4b-2006 navrhnutý v populárnom simulačnom nástroji OMNeT++. Mimo štruktúrovania jednotlivých uzlov podľa funkcionality na FFD a RFD, model implementuje podstatnú časť správy PAN siete tak, ako je definovaná v špecifikácii. V porovnaní s našim predchádzajúcim modelom siete IEEE 802.15.4-2003 definuje flexibilnejšiu štruktúru, je naviazaný na Mobility Framework a zároveň predpripravený pre nasadenie v prichádzajúcom simulačnom frameworku MiXiM. Mimo schopnosti vyhľadať sieť a asociovať sa do nej, pracuje model s CSMA-CA mechanizmom, vie si vytvárať a udržovať zoznamy susediacich prvkov a~fungujúcich PAN sietí vo svojom okolí, vysielať a spracovávať beacon rámce a má predpripravené základné konštrukcie pre nasadenie GTS mechanizmu.\\
\indent Práca ponúkla silný základ pre implementáciu IPv6 protokolu s použitím adaptačnej vrstvy. IP over IEEE predstavuje veľmi užitočnú konfiguráciu, ktorá rozširuje možnosti komunikácie založenej na IP protokole v zmiešaných prostrediach. Z dôvodu, že technológia je pomerne nová (predstavená v r. 2007) a zatiaľ nie je masívnejšie nasadzovaná, sú akékoľvek charakteristiky jej správania sa získané simuláciami veľmi cenné.\\
\indent V prípadnom pokračovaní a rozšírení tejto práce je priestor na dokončenie GTS mechanizmu, spracovanie osirotenia prvku, ktorý je úzko naviazaný na predstavené vlastnosti mobility nášho modelu a doplnenie vyšších vrstiev, hlavne sieťovej o~smerovanie rámcov. Po implementácii týchto vlastností by bolo prospešné doplniť model o~bezpečnostné procesy a pozorovať ich vplyv na prenosové a výkonové vlastnosti sietí ZigBee. Tiež sa ponúka využitie v simulátore MiXiM na testovanie vzájomného vplyvu IEEE 802.15.4 sietí so sieťami postavenými napríklad na nastupujúcom štandarde IEEE 802.11n.\\