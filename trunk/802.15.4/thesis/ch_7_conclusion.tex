\chapter{Záver}
\indent\indent V práci sme prezentovali simulačný model sietí ZigBee a IEEE 802.15.4b-2006 navrhnutý v populárnom simulačnom nástroji OMNeT++. Mimo štruktúrovania jednotlivých uzlov podľa funkcionality na FFD a RFD, model implementuje podstatnú časť správy PAN siete, ako je definovaná v špecifikácii. V porovnaní s našim predchádzajúcim modelom siete IEEE 802.15.4-2003 definuje flexibilnejšiu štruktúru, je naviazaný na Mobility Framework a zároveň predpripravený pre nasadenie v prichádzajúcom simulačnom frameworku MiXiM. Mimo schopnosti vyhľadať sieť a asociovať sa do nej, pracuje model s CSMA-CA mechanizmom, vie si vytvárať a udržovať zoznamy susediacich prvkov a fungujúcich PAN sietí vo svojom okolí, vysielať a spracovávať beacon rámce a má predpripravené základné konštrukcie pre nasadenie GTS mechanizmu.\\
\indent Predstavený bol aj model prenosu IPv6 paketov sieťami LR-WPAN. Takéto využitie by sa ponúkalo napríklad v PDA zariadeniach.\\
\indent V prípadnej budúcej práci je priestor na dokončenie daného GTS mechanizmu, spracovanie osirotenia prvku, ktorý je úzko naviazaný na predstavené vlastni mobility nášho modelu a doplnenie vyšších vrstiev, hlavne sieťovej o smerovanie rámcov. Po implementácii týchto vlastností by bolo prospešné doplniť model o bezpečnostné procesy a pozorovať ich vplyv na prenosové a výkonové vlastnosti sietí ZigBee. Tiež sa ponúka využitie v simulátore MiXiM na testovanie vzájomného vplyvu s WiFi sieťami postavenými napríklad na nastupujúcom štandarde IEEE 802.11n.\\