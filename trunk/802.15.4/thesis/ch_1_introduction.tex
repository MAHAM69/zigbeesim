\chapter{Úvod}

\indent\indent Od napísania našej predchádzajúcej práce zaoberajúcej sa simuláciami senzorových sietí~\cite{halas03} uplynuli približne tri roky. Technológia sa posunula o malý krok vpred a zariadenia pre komunikáciu v senzorových sieťach nenáročných na šírku dátového prenosu sa stávajú cenovo dostupnejšie. Vynárajú sa otázky, či pri masívnejších nasadeniach sú tieto siete schopné vykonávať požadované úlohy pri zdieľaní prenosového média spolu s takisto čoraz rozšírenejšími technológiami bezdrôtových sietí ako Wi-Fi\texttrademark, alebo Bluetooth\texttrademark, s ktorými zdieľajú časti frekvenčného spektra pre svoju komunikáciu.\\
\indent Na základe týchto skutočností sa ponúkalo zúročiť získané skúsenosti so simulačnými systémami a pripraviť aktuálnejší model siete ZigBee postavenej nad technológiou IEEE (Institute of Electrical and Electronics Engineers) 802.15.4\texttrademark, ktorý by reflektoval zmeny štandardov z uplynulých mesiacov a rokov a ktorý by ponúkal bohatšie možnosti simulácii s výsledkami bližšími reálnemu svetu. Medzičasom sa objavili možnosti prenosu paketov z rodiny TCP/IP (Transmission Control Protocol/Internet Protocol) v~senzorových sieťach, bolo by preto zaujímavé zistiť, aké výsledky podáva takáto konfigurácia.